\documentclass[12pt]{jreport}
\renewcommand{\bibname}{参考文献} 
\usepackage[dvipdfmx]{graphicx}
\usepackage{ascmac, url, style/moreverb,multirow,style/eclbkbox,fancybox,enumerate}

\setlength{\textheight}{25cm}		%1ページ当りの行数を指定する
\setlength{\textwidth}{38zw}		%1行あたりの文字数の設定
\setlength{\evensidemargin}{10mm}   %偶数ページの余白
\setlength{\oddsidemargin}{10mm}    %奇数ページの余白
\setlength{\topmargin}{-3mm}        %上の余白
\setlength{\headheight}{0mm}        %ヘッダ領域の高さ


\setlength{\headsep}{0mm}           %ヘッダ領域と本文領域との間隔
\setlength{\columnsep}{12mm}        %段組にした場合の段同士の間隔
\setlength{\footskip}{10mm}         %フッタ領域と本文との間隔
\setlength{\belowcaptionskip}{5pt}
%macro setting

%画像用マクロ
\newenvironment{img}[0]
{
\begin{figure}[!h]
\begin{screen}
\begin{center}
}{
\end{center}
\end{screen}
\end{figure}
}


%コマンド表示用 
\newenvironment{cmd}[0]
{
\begin{screen}
}{
\end{screen}
}

%スクリプト表示
\newenvironment{script}{
\VerbatimEnvironment
\begin{breakbox}
\small
\begin{Verbatim}
}{
\end{Verbatim}
\normalsize
\end{breakbox}
}					%マクロの読み込み


%%%%%%%%%%%%%%%%%%%%%%%%%%%%%%%%%%%%%%%%%%%%%%%%%%%%%%%%%%%%%%%%%%%%%%%

\title{AIを用いた楽曲制作に関する検討}					%卒業論文タイトル
\author{1532117 秋場  翼\\1532151 松元 孝樹\\\normalsize 指導教員:中村 直人 教授}	%名前(苗字と名前は全角1字空け)
\date{平成31年度}                   %日付設定(デフォルトは現在の年月日)

%%%%%%%%%%%%%%%%%%%%%%%%%%%%%%%%%%%%%%%%%%%%%%%%%%%%%%%%%%%%%%%%%%%%%%%

\begin{document}
\pagenumbering{roman}
\maketitle                        	%タイトルをドキュメントへ貼り付け
\tableofcontents               	%目次を作成
\listoffigures				%図の目次作成
\listoftables				%表の目次作成

\baselineskip 20pt              	%行間設定

\clearpage
\pagenumbering{arabic}


%ここから内容

%\input{ディレクトリ名/ファイル名}
\chapter{序論}

\section{研究の背景と目的}
スマートスピーカなどの対話型のAIが商品化され,現在ではスマートフォンにも搭載されるなどAIの存在は非常に身近になっている.また囲碁や将棋などの競技においても,プロに勝利するなどその精度は高くなっており,その発展は様々な分野にわたる.芸術の分野ではまだ発展途上ではあが,絵画や音楽に関してもAIを用いて
 近年,AI分野は急速な発展を続けている.スマートスピーカなどの対話型のAIがGoogleやAmazonによって商品化され,現在ではスマートフォンにも搭載されるなどその存在は非常に身近になっており,その種類も非常に多岐にわたる.
 また囲碁や将棋などの競技においても,プロに勝利するなどその精度は高くなっており,その成長は著しい.芸術の分野ではまだ発展途上ではあるが,絵画や音楽に関してもAIを用いて新しい作品を作るものが出回っている.\\
 このようにAIの発展は様々な分野においてその成果を上げており,今後は業務の効率化や補助だけにとどまらず,自動車の自動運転や医療の現場でも人間の手よりも高精度なものとして活躍することが期待されている.
 本研究ではAIによる楽曲生成についての実証実験を行う.Googole brainによって公開されているMagentaを用いて学習データやノード数による楽曲の生成結果の違いを検証し,チューリングテスト方式を用いてAIによる楽曲制作が有用なものか調査する.


\section{本論文の構成}
第1章では本論文の背景と目的について述べている.\\
第2章では本論文で利用する理論について述べている.\\
第3章では実験内容について述べている.\\
第4章では楽曲制作について述べている.\\
第5章では楽曲の有用性の調査について述べている.\\
第6章ではAIを用いた楽曲制作についての本研究の結論について述べている.\\
\chapter{理論}
\section{AIを用いた楽曲作成}

\subsection{MIDI}

 AIによる曲制作では主にMIDIファイルの音楽データを使用する.MIDI ファイルには実際の音ではなく音楽の演奏情報(音の高さや長さなど)である.
 本研究で用いるAIはこのMIDIファイルの情報を元に学習をする.また入出力の際もこの規格を用いる.

\subsection{Magenta}
 
 本研究ではMagenta[1]を使用する.これは音楽などをTensorFlowを使って機械学習するライブラリであり,Google BrainがGitHab上に公開している.
 Magentaではまず学習させたい音楽のMIDIデータをファイルに格納しNoteSequence(magentaが扱うファイル形式)に変更する.それを学習用データセットに変換したあと学習を行う.このとき,一度に学習させるデータの数,学習を行う回数,ノード数を設定する.これをパッケージ化し,MIDIファイルとして新たに楽曲を生成するという流れである.これを図2.1に示す.

\section{開発環境の構築}

 開発環境の構築にはコンテナ型仮想環境を提供するオープンソフトウェアであるDockerを用いた.\\
 Dockerには作成した仮想環境を配布可能な形にする事ができるDockerImageがあり,そのImageを用いる事で同一の実行環境が作成できる.
また,クラウド上でDockerImageを配布できるDockerHubというサービスがあり,そのサービス上にすでにMagentaの開発環境を構築済みの仮想環境があるため、その環境を今回は利用した.


\chapter{実験内容}
\section{学習回数による違い}

\section{ノード数による違い}

 AIによる曲制作では主にMIDIファイルの音楽データを使用する.MIDI ファイルには実際の音ではなく音楽の演奏情報(音の高さや長さなど)である.\\
 本研究で用いるAIはこのMIDIファイルの情報を元に学習をする.また入出力の際もこの規格を用いる.\\

\chapter{楽曲制作}
\section{MelodyRNN}

\subsection{BasicRNN}

\subsection{LookbackRNN}

\subsection{AttisionRNN}

\section{PolyfonyRNN}

\chapter{結論}

本研究では東京メトロが公開しているオープンデータを用いて,iPhoneとウェアラブル端末である,Apple Watchで使える,最新の遅延情報をリアルタイムに通知するアプリケーションの開発を行った.\\
 第 2 章では,開発環境である,東京メトロのオープンデータ,Swift,Xcode,Apple Watchの特徴について述べた.\\
 第 3 章は,本研究のアプリケーションのソースコード,またその用語,製作したアプリケーションの概要について述べた.\\
\newpage

\section{今後の課題}
aaa\\

\newpage

%ここまで内容


%%%%%%%%%%%%%%%%%%%%%%%%%%%%%%%%%%%%%%%%%%%%%%%%%%%%%%%%%%%%%%%%%%%%%%%

\chapter*{ \\謝辞}\addcontentsline{toc}{chapter}{謝辞}				%謝辞
%本研究に関しまして,終始懇切なるご指導いただきました,千葉工業大学 情報科学研究科 情報科学専攻 中村直人教授に心から御礼を申し上げます.情報科学ネットワーク学科の先生方,そして,中村研究室の皆様に心から感謝致します.皆様のおかげで,研究生活を充実かつ楽しく送ることが出来ました.最後に,これまで温かく見守りそして辛抱強く支援してくださいました,両親をはじめとする親族各位に改めて深い感謝の意を表して謝辞と致します.

%%%%%%%%%%%%%%%%%%%%%%%%%%%%%%%%%%%%%%%%%%%%%%%%%%%%%%%%%%%%%%%%%%%%%%%

\begin{thebibliography}{99}\addcontentsline{toc}{chapter}{参考文献}	%参考文献

%\bibitem{tankoubon} %キーワード:文章中で呼ぶ時に使う
%(単行本の場合)著者名(発行年)書名,発行所

%\bibitem{ronbun}
%(論文の場合)著者(発表年)タイトル,雑誌名,巻数,論文所在ページ

%\bibitem{webpage}
%(webページの場合)著者(発行年)表題.\\
%(改行して)http://・・・


%\bibitem{VM}
%大久保 健一,大塚 弘毅,染谷 文昭,照川 陽太郎“できるPROシリーズVMware vSphere6”.

\bibitem{metro}
東京メトロ,https://developer.tokyometroapp.jp/info .

\end{thebibliography}

%%%%%%%%%%%%%%%%%%%%%%%%%%%%%%%%%%%%%%%%%%%%%%%%%%%%%%%%%%%%%%%%%%%%%%%

\end{document}